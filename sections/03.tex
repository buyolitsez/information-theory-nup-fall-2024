%23.10.2024, lecture 3

\section{Coding}

\begin{definition}
    Let $\mathcal{A} = \{ a_{1}, a_{2}, \dots, a_{n} \}$ be an alphabet.
    A \emph{code} is a function $C: \mathcal{A} \to \{ 0, 1 \}^{\ast}$ which assigns codewords to the letters of the alphabet.
    If every message obtained by applying the code $C$ can be decoded uniquely, then the code is called uniquely decodable.
\end{definition}

We say that $c$ is a \emph{codeword} of the code $C$ if $c = C(a)$ for some $a \in \mathcal{A}$.

\begin{theorem} [Kraft-McMillan Inequality]
    \label{thm:Kraft-McMillan}
    For every uniquely decodable code with codewords $\{ c_{1}, c_{2}, \dots, c_{n} \}$, the following holds:
    \[
        \sum _{ i = 1 }^{ n } 2^{-|c_{i}|} \le 1.
    \]
\end{theorem}

\begin{proof}
    Idea.
    Let $p_{i}$ be a probability of $a_{i}$, where $p_{i} \approx 2^{-|c_{i}|} \implies |c_{i}| \approx - \log p_{i}$

    Let $\{ c_{1}, c_{2}, \dots, c_{n} \}$ be uniquely decodable

    \[
        \sum _{ i = 1 }^{ n } 2^{-|c_{i}|} \le 1
    \]

    Let's define a monomial for each code (assume that our ring is not commutative):
    If $c_{i} = 0101101$, then $p_{i}(x, y) = xyxyyxy$ and so on.
    Now, let's define a polynomial (for now, assume that $L$ is just some number):
    \begin{align*}
        f(x, y) &= \left( \sum _{ i = 1 }^{ n }  p_{i}(x, y) \right)^{L}  \\
        &= \sum _{ l = L }^{ L \cdot \max |c_{i}| } M_{l}(x, y)\text{, where } M_{l}(x, y) \text{is a sum of all monomials of degree } l.
    \end{align*}

    As codes are uniquely decodable, one can say that every monomial occurs in $M_{l}$ at most once.
    Hence $\# \{ \text{monomials of degree } l \} \le 2^{l}$.
    Thus we have:
    \[
        f\left( \frac{1}{2}, \frac{1}{2} \right) \leq \sum _{ l = L }^{ L \cdot \max |c_{i}| } 2^{l} \cdot \left( \frac{1}{2} \right)^{l} \le L \cdot \max |c_{i}| = O(L).
    \]
    Now, assume that $\sum_1^n 2^{-|c_{i}|} > 1 \implies \sum_1^n {2}^{-|c_{i}|}  = 1 + \varepsilon$ for some $\varepsilon > 0$.
    Note several facts:
    \begin{enumerate}
        \item $f\left( \frac{1}{2}, \frac{1}{2} \right) = O(L)$.
        \item $f\left( \frac{1}{2}, \frac{1}{2} \right) = \left( \sum_{i = 1}^n 2^{-|c_{i}|} \right)^{L} = (1 + \varepsilon)^{L}$ (since $p_{i}\left( \frac{1}{2}, \frac{1}{2} \right) = 2^{-|c_{i}|}$).
    \end{enumerate}
    Hence, we have a contradiction.
\end{proof}

\subsection{Prefix-Free Codes}

\begin{definition}
    A code is called prefix (prefix-free) if no codewords is a prefix of another codeword.
\end{definition}


\begin{theorem} [Kraft-McMillan Inequality for prefix codes]
    For prefix code with codewords $\{ c_{1}, c_{2}, \dots, c_{n} \}$:
    \[\sum_{i = 1}^{n}2^{-l_{i}} \le 1,\]
    where $l_i = |c_i|$.
\end{theorem}
\begin{proof}
    For each node in the tree at depth $k$ one can assign a probability $\frac{1}{2^{k}}$.
    Hence, for a leaf $c_i$ we will have the probability $2^{-l_i}$, but the sum over all leaves is 1.
    See~\Cref{fig:image_2024-10-23_22-13-28}.
    \begin{figure}[H]
        \centering
        \includegraphics[width=0.5\textwidth]{figures/image_2024-10-23_22-13-28}
        \caption{Probabilities on a prefix code.}
        \label{fig:image_2024-10-23_22-13-28}
    \end{figure}
\end{proof}

The reverse statement is also correct:
\begin{theorem}
    For a set of integers $\{ l_{1}, l_{2}, \dots, l_{n} \}$ that satisfies $\sum _{ i = 1 }^{ n } 2^{-l_{i}} \le 1$, there exists a prefix code with codewords $\{ c_{1}, c_{2}, \dots, c_{n} \}$ where $|c_{i}| = l_{i}$.
\end{theorem}
\begin{proof}
    At first, we create a binary tree with leaves at depths: $\{ l_{1}, l_{2}, \dots, l_{n} \}$ .
    Each leaf will be a codeword.

    Take a segment $[0, 1]$ and place each $2^{-l_i}$ on it in the desceding order.
    Start taking its parts greedily
    Then, you can construct . \todo{todo}
    $\text{Segment} \longleftrightarrow \text{Tree} \longleftrightarrow \text{codewords with} \sum _{ i = 1 }^{ n } 2^{-l_i} \le 1$
\end{proof}

\begin{corollary}
    For any unique decodable code, there exists a prefix code with the same length of codewords.
\end{corollary}

\subsubsection{Lower Bound}

\begin{theorem}[Shanon]
    For any distribution of letters $\alpha$ and any uniquely decodable code (so probability of symbol $a_i$ is $p_i$ and its code is $c_i$), the following holds:
    \[
        \text{Average code length} =  \sum _{ i = 1 }^{ n } p_{i} \cdot |c_{i}| \ge \sum _{ i = 1 }^{ n } p_{i} \cdot \log \frac{1}{p_{i}} = H(\alpha).
    \]
\end{theorem}

Since we have distribution $\alpha$, then we need at least $H(\alpha)$ bits to encode a symbol. \todo{not sure what i've just said}
\begin{proof}
    Move everything to the right side and apply Jensen's inequality:
    \begin{align*}
        \sum _{ i = 1 }^{ n } p_{i} \cdot \log \frac{2^{-|c_{i}|}}{p_{i}}
        &\le \log \sum _{ i = 1 }^{ n } \left( p_{i} \frac{2^{-|c_{i}|}}{p_{i}} \right)  \\
        &= \log \sum _{ i = 1 }^{ n } 2^{-|c_{i}|} \\
        &\le \log 1 = 0. \\
    \end{align*}
\end{proof}

\subsubsection{Upper Bound}

\begin{theorem}[Shannon]
    For every distribution $\alpha = \{ p_{1}, p_{2}, \dots, p_{n} \}$ there exists a prefix code $\{ c_{1}, c_{2}, \dots, c_{n} \}$:
    \[
        \sum _{ i = 1 }^{ n } p_{i} \cdot |c_{i}| \le \left(\sum _{ i = 1 }^{ n } p_{i} \cdot \log \frac{1}{p_{i}}\right) + 1 = H(\alpha) + 1.
    \]
\end{theorem}
\begin{proof}
    Let $l_{i} = \left\lceil  \log \frac{1}{p_{i}}  \right\rceil$.
    Note that $\{ l_{i} \}$ satisfy the Kraft-McMillan inequality~\ref{thm:Kraft-McMillan}:
    \begin{align*}
        \sum _{ i = 1 }^{ n } 2^{-|c_{i}|} &= \sum _{ i = 1 }^{ n } 2^{-\lceil \log 1/p_{i} \rceil }  \\
        &\le \sum _{ i = 1 }^{ n } 2^{-\log 1/p_{i}} \\
        &= \sum_{i =1}^{n} p_{i} \\
        &= 1.
    \end{align*}
    Let's estimate the average code length:
    \begin{align*}
        \sum _{ i = 1 }^{ n } p_{i} \cdot l_i &= \sum _{ i = 1 }^{ n } p_{i} \cdot \left\lceil  \log \frac{1}{p_{i}}  \right\rceil  \\
        &< \sum _{ i = 1 }^{ n } p_{i} \cdot \left( \log \frac{1}{p_{i}} + 1 \right) \\
        &= \left( \sum_{i = 1}^{n}p_{i} \cdot \log \frac{1}{p_{i}} \right) + 1.
    \end{align*}
\end{proof}

\subsubsection{Examples}

\begin{example}[Shannon-Fano Code]
    Assume that $p_{1} \ge p_{2} \ge \dots \ge p_{n}$, and assign it to the sub-segments of $[0, 1]$.
    All codes in $\left[ 0, \frac{1}{2} \right]$ starts with $0$, all others start with $1$. Continue recursively.
    \begin{figure}[H]
        \centering
        \includegraphics[width=0.5\textwidth]{figures/telegram-cloud-document-2-5244613392966114247}
        \caption{Shannon-Fano code.}
        \label{fig:telegram-cloud-document-2-5244613392966114247}
    \end{figure}
    Obviously, it is a prefix-free code
    See \Cref{fig:telegram-cloud-document-2-5244613392966114247}.
\end{example}

\begin{example}[Haffman Code]
    \todo[inline]{complete haffman code}
    Huffman is an optimal code.
    To prove that, one needs both Shannon theorems.
    Assume that $p_{1} \ge p_{2} \ge \dots \ge p_{n}$.
    Use greedy algorithm to build a Huffman Tree.
\end{example}

\subsubsection{Summary}

\begin{itemize}
    \item For every uniquely decodable code there is a prefix code with the same code length.
    \item The average code length of the optimal code is related to Shannon Entropy:
    \[
        H(\alpha) \le \sum _{ i = 1 }^{ n } p_{i} \cdot |c_{i}| \le H(\alpha) + 1.
    \]
    \item The $+1$ on the right side cannot be eliminated: for example, if we have only two symbols in the alphabet, then $\sum p_{i} \cdot |c_{i}| = 1$, while $\sum p_{i} \cdot \log \frac{1}{p_{i}}$ can be arbitrarily close to zero.
    For example, for any $\eps > 0$ there is a code with probabilities:

    \begin{center}
    \begin{tabular}{|c|c|c|}
        \hline
        $a_{i}$ & $p_{i}$ & $|c_{i}|$ \\
        \hline
        $a_{1}$ & $\varepsilon$ & 1 \\
        $a_{2}$ & $1 - \varepsilon$ & 1 \\
        \hline
    \end{tabular}
    \end{center}
\end{itemize}

\subsection{Block Coding}

Now we encode not individual symbols but blocks of symbols.
Let each block consist of $k$ symbols.
Let the random variables $\alpha_{1}, \alpha_{2}, \dots, \alpha_{k}$ be distributed as $\alpha$ and correspond to the letters in the block.
\[
H(\alpha_{1}, \alpha_{2}, \dots, \alpha_{k}) = \sum_{i = 1}^{k} H(\alpha_{i}) = k \cdot H(\alpha)
\]
Due to Shannon's theorems, we have the following:
\[
H(\alpha) \le \text{[average length of the letter code in the block]} \le H(\alpha) + \frac{1}{k}.
\]

Note that when coding blocks of length 100, we achieved a deviation from entropy of no more that $0.01$. However, we cannot apply the Huffman code since the algorithm for its construction \todo{finish}

\subsubsection{Arithmetic Coding}

\todo[inline]{no idea}

\begin{definition}
    Assume that $p_{1} \ge p_{2} \ge \dots \ge p_{n}$ and assign it to the subsegements of $[0, 1]$ (all numbers below are base 2 and we omit $0.$ at their beginning, since all of them in $[0, 1)$).
    We say that half-open interval is \emph{standard} if it has the form $[v 0, v 1]$, where $v \in \{ 0, 1 \}^{\ast}$.
    We will assign to each standard interval $[v 0, v1]$ the code $v0$ю
\end{definition}

\begin{statement}
    Arithmetic coding is not as optimal.
\end{statement}
\begin{proof}
    \[\sum _{ i = 1 }^{ n } p_{i} \cdot |c_{i}| \le \sum _{ i = 1 }^{ n } p_{i} \cdot \log \frac{1}{p_{i}} + 2. \qedhere \]
\end{proof}

\subsubsection{Block Coding With Errors}

Let $\alpha_{1}, \alpha_{2}, \dots, \alpha_{n}$ be i.i.d. random variables on $\{ a_{1}, a_{2}, \dots, a_{k} \}$ with probabilities $p_{1}, p_{2}, \dots, p_{k}$.
Consider block coding defined by functions $E_{n}$ and $D_{n}$:
\begin{itemize}
    \item $E_{n}: \{ a_{1}, a_{2}, \dots, a_{k} \}^{n} \to \{ 0, 1 \}^{L_{n}}$.
    \item $D_{n}: \{ 0, 1 \}^{L_{n}} \to \{ a_{1}, \dots, a_{k} \}^{n}$.
\end{itemize}

The error probability $\varepsilon_{n}$ is the probability of the following event:
\[
\left[ (\alpha_{1}, \alpha_{2}, \dots, \alpha_{n}) = (a_{i_{1}}, a_{i_{2}}, \dots, a_{i_{n}}) \mid D_{n}(E_{n}(a_{i_{1}}a_{i_{2}}, \dots, a_{i_{n}})) \ne (a_{i_{1}}, a_{i_{2}}, \dots, a_{i_{n}}) \right].
\]


